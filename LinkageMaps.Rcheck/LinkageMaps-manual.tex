\nonstopmode{}
\documentclass[a4paper]{book}
\usepackage[times,inconsolata,hyper]{Rd}
\usepackage{makeidx}
\usepackage[utf8,latin1]{inputenc}
% \usepackage{graphicx} % @USE GRAPHICX@
\makeindex{}
\begin{document}
\chapter*{}
\begin{center}
{\textbf{\huge Package `LinkageMaps'}}
\par\bigskip{\large \today}
\end{center}
\begin{description}
\raggedright{}
\item[Type]\AsIs{Package}
\item[Title]\AsIs{Linkage Maps}
\item[Version]\AsIs{0.1}
\item[Date]\AsIs{2012-08-23}
\item[Author]\AsIs{Michael Sieger}
\item[Suggests]\AsIs{MASS}
\item[Depends]\AsIs{aplpack, cluster}
\item[Maintainer]\AsIs{Michael Sieger }\email{michael.sieger@student.hswt.de}\AsIs{}
\item[Description]\AsIs{The Linkage Maps package provides functions to display multivariate data, based on sammons nonlinear mapping.}
\item[URL]\AsIs{}\url{http://www.hswt.de/fh/fakultaet/gl/dozenten/ohmayer.html}\AsIs{}
\item[License]\AsIs{LGPL}
\end{description}
\Rdcontents{\R{} topics documented:}
\inputencoding{utf8}
\HeaderA{fuzzyColorPlot}{Displays a fuzzy color plot}{fuzzyColorPlot}
%
\begin{Description}\relax
A fuzzy color plot is a 2D scatterplot that displays the result of sammons nonlinear mapping. Additionally a fuzzy clustering is made.

If the slider is on the first position, all cluster are displayed together. 
The color/symbol indicates the nearest crisp clustering then. 
The color intensity displays the probability of membership to the nearest cluster.

You can select a single cluster too. The color intensity shows the probability of membership to the selected cluster.


\end{Description}
%
\begin{Usage}
\begin{verbatim}
	fuzzyColorPlot(X, k, Xs, clusterColors=rainbow(k), clusterSymbols=rep(21,k), xlab="", ylab="", main="")
\end{verbatim}
\end{Usage}
%
\begin{Arguments}
\begin{ldescription}

\item[\code{X}] 
A matrix with size (nPoints,nDimensions) that contains the untransformed input data.


\item[\code{k}] 
The number of clusters.


\item[\code{Xs}] 
A matrix with size (nPoints,2) that contains the projected points from sammons nonlinear mapping.


\item[\code{clusterColors}] 
A vector of size k that contains the colors for the clusters. You should only choose colors with a very high intensity.


\item[\code{clusterSymbols}] 
A vector of size k that contains the symbols for the clusters. Not all symbols can be filled. If you want this you should use the symbols 15-20.


\item[\code{xlab}] 
As described in \LinkA{plot}{plot}.

\item[\code{ylab}] 
As described in \LinkA{plot}{plot}.

\item[\code{main}] 
As described in \LinkA{plot}{plot}.

\end{ldescription}
\end{Arguments}
%
\begin{Author}\relax
Michael Sieger <michael.sieger@student.hswt.de>
\end{Author}
%
\begin{Examples}
\begin{ExampleCode}

# The example data
data("Milch")

library(cluster)
library(MASS)

D = dist(Milch[3:6])
HK <- princomp(Milch[3:6], cor=FALSE, scores=TRUE)
S <- sammon(D, HK$scores[,1:2])

# A very basic example showing the default colors an symbols
fuzzyColorPlot(Milch[3:6], 5, S$points)

# Custom symbols 
## Not run: fuzzyColorPlot(Milch[3:6], 5, S$points, clusterSymbols=c("a","b", "c","d", "e"))

# A Black-White Plot
## Not run: fuzzyColorPlot(Milch[3:6], 5, S$points, clusterColors=rep("black", 5), clusterSymbols=15:19)


\end{ExampleCode}
\end{Examples}
\inputencoding{utf8}
\HeaderA{linkmap}{Displays a Linkage Map}{linkmap}
%
\begin{Description}\relax
A Linkage Map is a 2D scatter plot that displays the result of sammons nonlinear mapping. You define a number (possibly one) of ranges. Each point pair whose distance
is inside of the defined range is connected with a line. You can change the ranges dynamically
at runtime with sliders.
\end{Description}
%
\begin{Usage}
\begin{verbatim}
linkmap(X, D=dist(X), linetypes=c("solid","dotted"), linecolors=c("red","green"), linewidths=c(1,1),
						 labels = NULL, cluster = NULL, maxValue=0.33, legendDigits = 2, xlab = "", ylab = "", main = "")
\end{verbatim}
\end{Usage}
%
\begin{Arguments}
\begin{ldescription}
\item[\code{X}] 
A matrix with size (n,2) that contains the projected points from sammons nonlinear mapping.

\item[\code{D}] 
A distance matrix for the X argument. 

\item[\code{linetypes}] 
An array of line types. The size must match with line colors and line widths.
The various line types are described in \LinkA{par}{par}

\item[\code{linecolors}] 
An array of colors. The size must match with line types and line widths.

\item[\code{linewidths}] 
An array of line widths. The size must match with line colors and line types.

\item[\code{labels}] 
A string-array with labels for the points

\item[\code{cluster}] 
A array containing cluster memberships of the points. The point
membership will be indicated with different colors. The array size
must match with the number of points.

\item[\code{maxValue}] 
The maximum value that can be adjusted with the sliders.

\item[\code{legendDigits}] 
The number of fractional digits to be displayed in the legend.

\item[\code{xlab}] 
As described in \LinkA{plot}{plot}.

\item[\code{ylab}] 
As described in \LinkA{plot}{plot}.

\item[\code{main}] 
As described in \LinkA{plot}{plot}.

\end{ldescription}
\end{Arguments}
%
\begin{Author}\relax
Michael Sieger <michael.sieger@student.hswt.de>
\end{Author}
%
\begin{Examples}
\begin{ExampleCode}

library(MASS)

data("Milch3")

D <- dist(Milch3[3:6])
HK <- princomp(Milch3[3:6], cor=FALSE, scores=TRUE)
S <- sammon(D, HK$scores[,1:2])
k <- kmeans(D,center=5)
linkmap(S$points, D, cluster=k$cluster,main="Milch")

\end{ExampleCode}
\end{Examples}
\inputencoding{utf8}
\HeaderA{Milch}{Milk components of mammals}{Milch}
\keyword{datasets}{Milch}
%
\begin{Description}\relax
This dataset contains the fat, protein, lactose and ash of different
milks.
\end{Description}
%
\begin{Usage}
\begin{verbatim}
Milch
\end{verbatim}
\end{Usage}
\inputencoding{utf8}
\HeaderA{Milch3}{Milk components of mammals}{Milch3}
\keyword{datasets}{Milch3}
%
\begin{Description}\relax
A smaller version of the Milch dataset
\end{Description}
%
\begin{Usage}
\begin{verbatim}
Milch3
\end{verbatim}
\end{Usage}
\printindex{}
\end{document}
