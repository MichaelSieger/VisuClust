\nonstopmode{}
\documentclass[letterpaper]{book}
\usepackage[times,inconsolata,hyper]{Rd}
\usepackage{makeidx}
\usepackage[utf8,latin1]{inputenc}
% \usepackage{graphicx} % @USE GRAPHICX@
\makeindex{}
\begin{document}
\chapter*{}
\begin{center}
{\textbf{\huge Package `linkmaps'}}
\par\bigskip{\large \today}
\end{center}
\begin{description}
\raggedright{}
\item[Type]\AsIs{Package}
\item[Title]\AsIs{Linkage Maps}
\item[Version]\AsIs{0.1}
\item[Date]\AsIs{2012-08-23}
\item[Author]\AsIs{Michael Sieger}
\item[Depends]\AsIs{aplpack}
\item[Suggests]\AsIs{MASS}
\item[Maintainer]\AsIs{Michael Sieger }\email{michael.sieger@student.hswt.de}\AsIs{}
\item[Description]\AsIs{Linkage Maps provide a way to display multivariate data. They are based on sammons-nonlinear mapping and connect the datapoints with lines depending on the real distance.}
\item[URL]\AsIs{}\url{http://www.hswt.de/}\AsIs{}
\item[License]\AsIs{What license is it under?}
\end{description}
\Rdcontents{\R{} topics documented:}
\inputencoding{utf8}
\HeaderA{linkmap}{Displays a linkage map}{linkmap}
%
\begin{Description}\relax
A Linkage Map is a 2D scatter plot that displays the result of sammons nonlinear mapping. You define a number (possibly one) of ranges. Each point pair whose distance
is inside of the defined range is connected with a line. You can change the ranges dynamically
at runtime with sliders.
\end{Description}
%
\begin{Usage}
\begin{verbatim}
linkmap(X, D=as.matrix(dist(X)), linetypes=c("solid","dotted"), linecolors=c("red","green"), linewidths=c(1,1),
						 labels = NULL, cluster = NULL, maxValue=0.33, legendDigits = 2, ...)
\end{verbatim}
\end{Usage}
%
\begin{Arguments}
\begin{ldescription}
\item[\code{X}] 
A matrix with size (n,2) that contains the projected points from sammons nonlinear mapping

\item[\code{D}] 
A distance matrix for the X argument. The size must be (n,n)

\item[\code{linetypes}] 
An array of line types. The size must match with line colors and line widths.
The various line types a descriped in \LinkA{par}{par}

\item[\code{linecolors}] 
An array of colors. The size must match with line colors and line widths.

\item[\code{linewidths}] 
An array of line widths. The size must match with line colors and line widths.

\item[\code{labels}] 
A string-array with labels for the points

\item[\code{cluster}] 
A array containing cluster memberships of the points. The point
membership will be indicated with different colors. The array size
must match with the number of points.

\item[\code{maxValue}] 
The maximum value that can be adjusted with the sliders

\item[\code{legendDigits}] 
The number of fractional digits to be displayed in the legend

\item[\code{...}] 
You may use arguments specified by \LinkA{plot}{plot}

\end{ldescription}
\end{Arguments}
%
\begin{Author}\relax
Michael Sieger <michael.sieger@student.hswt.de>
\end{Author}
%
\begin{Examples}
\begin{ExampleCode}

library(MASS)

MD <- read.table(text="
Row  Name              Fett   Eiweiss Laktose Asche
  1  Mensch                  3,8      1,0       7,0      0,2
  2  Orang-Utan              3,5      1,5       6,0      0,2
  3  Schimpanse              3,7      1,2       7,0      0,2
  4  Zwergmeerkatze          2,9      2,1       7,2      0,3
  5  Pavian                  5,0      1,6       7,3      0,3
  6  Tamarin                 3,1      3,8       5,8      0,4
  7  Esel                    1,4      2,0       7,4      0,5
  8  Hauspferd               1,9      2,5       6,2      0,5
  9  Wildpferd               2,2      2,0       6,1      0,4
 10  Zebra                   2,1      2,3       8,3      0,4
 11  Wildschwein             6,8      4,8       5,5      1,7
 12  Lama                    2,4      7,3       6,0      0,5
 13  Kamel                   5,4      3,9       5,1      0,7
 14  Dromedar                4,5      3,6       5,0      0,7
 15  Sikahirsch             19,0     12,4       3,4      1,4
 16  Rothirsch              19,7     10,6       2,6      1,4
 17  Ren                    20,0      9,5       2,6      1,4
 18  Edmigazelle            19,0     12,4       3,3      1,5
 19  Thompson-Gazelle       19,6     10,5       2,7      1,4
 20  Schwarzfersenantilope  20,4     10,8       2,4      1,4
 21  Hausrind                3,7      3,4       4,8      0,7
 22  Zebu                    4,7      3,2       4,9      0,7
 23  Yak                     6,5      5,8       4,6      0,9
 24  Wasserbueffel            7,4      3,8       4,8      0,8
 25  Bison                   3,5      4,5       5,1      0,8
 26  Moschusochse            5,4      5,3       4,1      1,1
 27  Hausziege               4,5      2,9       4,1      0,8
 28  Hausschaf               7,4      5,5       4,8      1,0
 29  Haushund               12,9      7,9       3,1      1,2
 30  Wolf                    9,6      9,2       3,4      1,2
 31  Kojote                 10,7      9,9       3,0      0,9
 32  Schakal                10,5     10,0       3,0      1,2
 33  Afrikan.Wildhund        9,5      9,3       3,5      1,3
 34  Schwarzbaer             24,5     14,5       0,4      1,8
 35  Grizzly-Baer            22,3     11,1       0,6      1,5
 36  Braunbaer               22,6      7,9       2,1      1,4
 37  Eisbaer                 33,1     10,9       0,3      1,4
 38  Nördlicher-Seebaer      53,3      8,9       0,1      0,5
 39  Biber                  11,7      8,1       2,6      1,1
 40  Goldhamster             4,9      9,4       4,9      1,4
 41  Wanderratte            10,3      8,4       2,6      1,3
 42  Hausmaus               13,1      9,0       3,0      1,3
 43  Hauskaninchen          18,3     13,9       2,1      1,8
 44  Florida-Waldkaninchen  13,9     23,7       1,7      1,5
 45  Wildkaninchen          17,9     12,5       1,0      2,0
 46  Blauwal                42,3     10,9       1,3      1,4
 47  Finnwal                32,4     17,8       0,3      1,0
 48  Buckelwal              33,0     12,5       1,1      1,6
 49  Delphin                33,0      6,8       1,1      0,7
 50  Braunbrustigel         10,1      7,2       2,0      2,3
", dec=",", header=TRUE)

D <- as.matrix(dist(MD[3:6]))
HK <- princomp(MD[3:6], cor=FALSE, scores=TRUE)
S <- sammon(D, HK$scores[,1:2])
k <- kmeans(D,center=5)
linkmap(S$points, D, cluster=k$cluster,main="Milch")

\end{ExampleCode}
\end{Examples}
\printindex{}
\end{document}
